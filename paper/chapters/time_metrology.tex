\chapter[Time metrology]


%====================================================================================================
\section{History and overview of metrologoy}

%----------------------------------------------------------------------------------------------------
\subsection{Metrology and reference values}

%----------------------------------------------------------------------------------------------------
\subsection{Time metrology through ages}

%----------------------------------------------------------------------------------------------------
\subsection{Modern approach with fous on GNSS related standards}
In modern day all measurement systems used in science and egineering are governed by international
standars. One of most well known set of standards are those provided by the International 
Organization for Standardization (IOS) \textbf{CITATION}.
In case of GNSS systems a standarized spatial and temporal frame must be decided on in order
for all element of system to work correctly.

\paragraph{Spatial reference frames}
In case of spatial information GNSS use two reference frames\textbf{CITATION}:
\begin{itemize}
	\item Earth-centered internal (ECI) for satellites
	\item Earth-centered Earth-fixed (ECEF) for ground recievers.
\end{itemize}
\textbf{SOME ECI RELATED IMAGE}
In case of satellites ECI is used as it is default reference frame for modelling spacecraft
motion \textbf{CITATION}. However in case of ground recievers that that are either fixed to 
specific point on earth or are moving from one geographical location to another ECEF is much
more natural way of describing position.
Reference frames for spatial data in GNSS systems are provief by:
\begin{itemize}
	\item International Terrestial Reference System (ITRS)
	\item International Celestial Reference System (ICRF)
\end{itemize}

\paragraph{Temporal reference frames}
When talking about time measurement reference frames in context of spaceborn objects a basic
concept from general relativity theory must be introduced.
One of those concept is \emph{proper time (PT)} which \textbf{WRITE THIS}.
Another is \emph{coordinate time (CT)} which denotes a perfect clock residing in a specific 
internal coordinate system \textbf{CITATION}.
In real life measurements CT is highest quality time measurement that we can achieve as a 
physical clock must reside in a physical space which by itself binds it to a coordinate system.
\textbf{WRITE ABOUT GENERAL TIME REFERENCE FRAMES}

\paragraph{Temporal reference frames in GNSS systems}


%====================================================================================================
\section{Mathematical clock model}

%----------------------------------------------------------------------------------------------------
\subsection{Discrete and continous clock models}
A mahematical model of a clock is a description of the bias in a sequence of time measurements.
Bias is a difference between the clock time, denoted as $T_{c}$, and the reference time,
denoted as a $T_{r}$. Both clock time and refeence time are going to be indexed with either
measurement number $i$ for a discrete model or a measurement timestamp  $t$ for a continous 
model. In case of continous model $T_{r}(t)=t$ as the reference clock is considered a
time source in model. Therefore bias can be described as a:
\begin{equation}
	b_{c}(i) = T_{r}(i) - T_{c}(i),
\end{equation}
for discrete model or a \textbf{FIX CONTINOUS EQUATION}
\begin{equation}
	b_{c}(t) = T_{r}(t) + \int_{t_0}^{t_1}T_{c}(\tau),
\end{equation}
for continous model.
In this work only discrete model in any furhter descriptions a term \emph{clock model} and
\emph{discreete clock model} will be used interchangably.



%----------------------------------------------------------------------------------------------------
\subsection{Basic clock model}

%----------------------------------------------------------------------------------------------------
\subsection{Oscilator noise model}

%----------------------------------------------------------------------------------------------------
\subsection{Standard clock model}


%====================================================================================================
\section{Stability analysis}

%----------------------------------------------------------------------------------------------------
\subsection{Allan variance}

%----------------------------------------------------------------------------------------------------
\subsection{Hadamard variance}

%----------------------------------------------------------------------------------------------------
\subsection{Sigma-tau curves}



%====================================================================================================
\section{Spaceborn clocks}

%----------------------------------------------------------------------------------------------------
\subsection{Clocks used in satellites}

%----------------------------------------------------------------------------------------------------
\subsection{Relativistic time transformation}

