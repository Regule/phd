\documentclass{kybernetika}
%the class kybernetika includes these packages:
% graphicx, amssymb, amsmath

%--------------------------------------------------------------------------------------------------
% used environment for theorems:

\newtheorem{theorem}{Theorem}[section]
\newtheorem{lemma}[theorem]{Lemma}
\newtheorem{proposition}[theorem]{Proposition}
\newtheorem{corollary}[theorem]{Corollary}
\newtheorem{remark}[theorem]{Remark}
\newtheorem{fact}[theorem]{Fact}
\newtheorem{example}[theorem]{Example}
\newtheorem{definition}[theorem]{Definition}
\newtheorem{observation}[theorem]{Observation}

\begin{document}

%==================================================================================================
% TITLE PAGE 
%==================================================================================================
\pagestyle{myheadings}
\title{Application of Long Short Term Memory neural networks for GPS satellite clock 
bias prediction}

\author{Piotr Gny\'{s}, Pawe\l{} Przestrzelski}

\contact{Piotr}{Gny\'{s}}
{Department of Computer Science,  Polish-Japanese Academy of Information Technology,
Koszykowa 86 Street, 02-008 Warsaw}{pgnys@pjwstk.edu.pl}

\contact{Pawe\l{}}{Przestrzelski}
{Department of Computer Science,  Polish-Japanese Academy of Information Technology,
Koszykowa 86 Street, 02-008 Warsaw}{pprzestrzelski@pjwstk.edu.pl}

\markboth{Piotr Gny\'{s}, Pawe\l{} Przestrzelski}{LSTM networks for GPS clock bias prediction}

\maketitle

\begin{abstract}
In this article the results of an application of Long Short Term Memory neural networks for
clock bias will be presented and compared against current state of the art in real-time
GPS clock bias prediction. Presented approach is intended to better quality of localization
for marine robots that operate autonomously for long periods of time away from convenient  
internet uplink.
\end{abstract}

\keywords{LSTM, Long Short Term Memory, Neural Networks, GPS, Navigation, Time series prediction}

\classification{68T05, 68T10, 68T40}
\end{document}
