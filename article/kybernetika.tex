\documentclass{kybernetika}
%the class kybernetika includes these packages:
% graphicx, amssymb, amsmath

%--------------------------------------------------------------------------------------------------
% used environment for theorems:

\newtheorem{theorem}{Theorem}[section]
\newtheorem{lemma}[theorem]{Lemma}
\newtheorem{proposition}[theorem]{Proposition}
\newtheorem{corollary}[theorem]{Corollary}
\newtheorem{remark}[theorem]{Remark}
\newtheorem{fact}[theorem]{Fact}
\newtheorem{example}[theorem]{Example}
\newtheorem{definition}[theorem]{Definition}
\newtheorem{observation}[theorem]{Observation}

\begin{document}

%==================================================================================================
% TITLE PAGE 
%==================================================================================================
\pagestyle{myheadings}
\title{Application of Long Short Term Memory neural networks for GPS satellite clock 
bias prediction}

\author{Piotr Gny\'{s}, Pawe\l{} Przestrzelski}

\contact{Piotr}{Gny\'{s}}
{Department of Computer Science,  Polish-Japanese Academy of Information Technology,
Koszykowa 86 Street, 02-008 Warsaw}{pgnys@pjwstk.edu.pl}

\contact{Pawe\l{}}{Przestrzelski}
{Department of Computer Science,  Polish-Japanese Academy of Information Technology,
Koszykowa 86 Street, 02-008 Warsaw}{pprzestrzelski@pjwstk.edu.pl}

\markboth{Piotr Gny\'{s}, Pawe\l{} Przestrzelski}{LSTM networks for GPS clock bias prediction}

\maketitle

\begin{abstract}
In this article the results of an application of Long Short Term Memory neural networks for
clock bias will be presented and compared against current state of the art in real-time
GPS clock bias prediction. Presented approach is intended to better quality of localization
for marine robots that operate autonomously for long periods of time away from convenient  
internet uplink.
\end{abstract}

\keywords{LSTM, Long Short Term Memory, Neural Networks, GPS, Navigation, Time series prediction}

\classification{68T05, 68T10, 68T40}


%==================================================================================================
\section{Introduction}
Research presented in this paper is part of a larger project that aims at developing a long
term autonomous marine robotics systems. However as Global Positioning System (GPS) is used
in many application elements specific to one that was intended when working on algorithm will
be omitted with exception of MOTIVATIONS section.

%--------------------------------------------------------------------------------------------------
\subsection{Motivation}
In recent years there was a rise in popularity of autonomous platforms. Examples include, but
are not limited to, Tesla self driving cars, autonomous container ships or robotic defence 
systems. All those applications have one thing in common, they require a precise navigation 
capabilities that will allow them to move without damaging themselves or their environments. 
This is especially challenging in marine environment as not only there is almost no visual
reference points on open see but also because any contact with external data source is expensive
as very often only connection available is satellite uplink.
In case of large vessels, like already mentioned autonomous container ships, this is only
problem of bandwidth cost. 
However in case of small robots like gliders or sailboats that rely on low power consumption 
any localization system is limited by low computation power and high energy consumption of 
satellite communication that discourages it frequent use.
This is why an effort was made to deploy a solution that provides a on-board solution for high
quality GPS based localisation with low computational overhead.

\end{document}
