\chapter{Machine learning based approach}


%====================================================================================================
\section{Concepts in machine learning}
Aim of this section is to define what exactly is understood by a term machine learning as well
as to introduce basic concepts that are common for all machine learning applications

%----------------------------------------------------------------------------------------------------
\subsection{Differences between knowledge and data based models}
While creating a model of physical processes one of two approaches may be used.
First, is the knowledge-driven model that as the name implies relays on knowledge of underlying
laws of physics that govern the process.
The main advantage of this approach is its high explainability and reliability as every parameter
of the model corresponds with some specific physical property.
However this model has a very significant downside, it requires detailed knowledge of physical
phenomena. This means that for complex processes it might be very difficult or even
impossible, at a~given moment, to create a model. Another drawback is that the complexity of
the process grows the computational cost of its model.
If those problem makes creating usable knowledge-based model impossible or economically
non efficient a data-driven approach may be used.
It model used is chosen arbitrarily with little or no relation to underlying physics,
then model parameters are adjusted until it will fit experimental data to the best of its
capability.
The main advantage of a data-driven approach is the lack of required knowledge on process 
inner workings, which allow prediction of events which exact mechanics have not 
yet been discovered.
Another element characteristic in the data-driven approach, that can be either advantageous or
disadvantageous, is the lack of direct correlation between the complexity 
of the process and computational cost of the model.
This is a very welcome trait for complex processes like orbital atomic clock ensembles.
The main disadvantage in comparison with the knowledge-driven approach is its lower reliability and
explainability. This is the main reason why the data-driven approach is usually avoided in
critical implementations. Another significant downside is a requirement of a large amount of
experimental data for model adjustment which means that in many cases this approach simply
cannot be used.
The last thing that must be said about the data-driven approach is that it still requires some
amount of knowledge to work efficiently as the selection of models that will be adjusted to
data requires some assumption. For example, the use of linear regression assumes that the model
is linear.

%----------------------------------------------------------------------------------------------------
\subsection{What is a machine learning algorithm}
Term machine learning describes subset of algorithms that adjusts parameter of other algorithms 
according to data. To provide more formal definition an algorithm can be described as a 
transformation function $\phi_{a}$ such that:
\begin{equation}
	\label{equ:algorithm_general}
	\mathcal{Y}_{a} = \phi_{a}(\mathcal{X},\Theta_{a}),
\end{equation}
where $\mathcal{Y}_{a}$ is set of algorithm responses, $\mathcal{X}$ is set of inputs 
and $\Theta_{a}$ is algorithm parameter set.
Input space $\mathcal{U}_{x}$ and response space $\mathcal{U}_{y}$ can be simply a numerical
space, in which case algorithm is called numerical algorithm, but can also represent more 
abstract concepts. For example in case of database software input space will consist of 
queries while response space will contain information stored in database as well as error 
status in case query was written incorrectly.
Machine learning algorithm, denoted as $\phi_{m}$ can also be described in that general form
however what is special in this case is that input space $\mathcal{U}_{m}$ consists of either :
\begin{equation}
	\label{equ:supervised_input}
	\mathcal{X}_{m} = \{\phi_{a}, \mathcal{X}_{a}, \mathcal{Y}_{a}, \Theta_{a} \},
\end{equation}
in case of supervised learning or just :
\begin{equation}
	\label{equ:supervised_input}
	\mathcal{X}_{m} = \{\phi_{a}, \mathcal{X}_{a}, \Theta_{a} \},
\end{equation}
in case of unsupervised learning.
In both cases response of machine learning algorithm is described as a:
\begin{equation}
	\label{equ:ml_response}
	\mathcal{Y}_{m} = \{\hat{\Theta}_{a}, E_{a}\},
\end{equation}
where $\hat{\Theta}_{a}$ is adjusted set of parameters and $E_{a}$ is value prediction 
error for this new set of parameters.
Aim of machine learning algorithm is to adjust parameters $\Theta_{a}$ of given algorithm 
$\phi_{a}$ in such a way to minimize error $E_{a}$ for given data. In case of supervised learning
a correct response $\mathcal{Y}_{a}$ for inputs is known where in case of unsupervised learning
only input set $\mathcal{X}_{a}$ is known. In order to avoid confusion parameters of machine 
learning algorithm $\Theta_{m}$ will be referred to as a \textit{metaparameters}. 
Exact set of meteaparameters vary between specific algorithms, however there is one 
crucial metaparameter that will appear in every machine learning algorithm.
That metaparameter is error function as it is required for error value calculation and as 
minimization of that value is goal of machine learning it is mandatory to be able to calculate it.

%----------------------------------------------------------------------------------------------------
\subsection{Error function}

%----------------------------------------------------------------------------------------------------
\subsection{Deep learning}

%====================================================================================================
\section{Regression approximation}

%----------------------------------------------------------------------------------------------------
\subsection{Linear regression}

%----------------------------------------------------------------------------------------------------
\subsection{Polynomial regression}

%----------------------------------------------------------------------------------------------------
\subsection{Support vector machine}


%====================================================================================================
\section{Gradient based optimization}

%====================================================================================================
\section{Neural networks}

%----------------------------------------------------------------------------------------------------
\subsection{Feed forward neural networks}

%----------------------------------------------------------------------------------------------------
\subsection{Simple recurrent neural networks}

%----------------------------------------------------------------------------------------------------
\subsection{Networks with long term memory}

