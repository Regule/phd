\chapter{Machine learning based approach}


%====================================================================================================
\section{Concepts in machine learning}

%----------------------------------------------------------------------------------------------------
\subsection{Differences between knowledge and data based models}
While creating a model of physical processes one of two approaches may be used.
First, is the knowledge-driven model that as the name implies relays on knowledge of underlying
laws of physics that govern the process.
The main advantage of this approach is its high explainability and reliability as every parameter
of the model corresponds with some specific physical property.
However this model has a very significant downside, it requires detailed knowledge of physical
phenomena. This means that for complex processes it might be very difficult or even
impossible, at a~given moment, to create a model. Another drawback is that the complexity of
the process grows the computational cost of its model.
If those problem makes creating usable knowledge-based model impossible or economically
nonefficient a data-driven approach may be used.
It model used is chosen arbitrarily with little or no relation to underlying physics,
then model parameters are adjusted until it will fit experimental data to the best of its
capability.
The main advantage of a data-driven approach is the lack of required knowledge on process 
inner workings, which allow prediction of events which exact mechanics have not 
yet been discovered.
Another element characteristic in the data-driven approach, that can be either advantageous or
disadvantageous, is the lack of direct correlation between the complexity 
of the process and computational cost of the model.
This is a very welcome trait for complex processes like orbital atomic clock ensembles.
The main disadvantage in comparison with the knowledge-driven approach is its lower reliability and
explainability. This is the main reason why the data-driven approach is usually avoided in
critical implementations. Another significant downside is a requirement of a large amount of
experimental data for model adjustment which means that in many cases this approach simply
cannot be used.
The last thing that must be said about the data-driven approach is that it still requires some
amount of knowledge to work efficiently as the selection of models that will be adjusted to
data requires some assumption. For example, the use of linear regression assumes that the model
is linear. The field of engineering that concerns itself with algorithms that automatically
adjust models to data is called Machine Learning (ML).


%----------------------------------------------------------------------------------------------------
\subsection{What is a machine learning algorithm}

%----------------------------------------------------------------------------------------------------
\subsection{Error function}

%----------------------------------------------------------------------------------------------------
\subsection{Deep learning}

%====================================================================================================
\section{Regression approximation}

%----------------------------------------------------------------------------------------------------
\subsection{Linear regression}

%----------------------------------------------------------------------------------------------------
\subsection{Polynomial regression}

%----------------------------------------------------------------------------------------------------
\subsection{Support vector machine}


%====================================================================================================
\section{Gradient based optimization}

%====================================================================================================
\section{Neural networks}

%----------------------------------------------------------------------------------------------------
\subsection{Feed forward neural networks}

%----------------------------------------------------------------------------------------------------
\subsection{Simple recurrent neural networks}

%----------------------------------------------------------------------------------------------------
\subsection{Networks with long term memory}

