\chapter{Evolutionary algorithms}


%====================================================================================================
\section{Biological evolution}

%--------------------------------------------------------------------------------------------------
\subsection{Dawrwin theory and its modern form}
The scientific theory of evolution by natural selection was conceived independently by
Charles Darwin and Alfred Russel Wallace in the mid-19th century and was set out in detail in
Darwin's book On the Origin of Species.
Evolution by natural selection was first demonstrated by the observation that more offspring are
often produced than can possibly survive. 
This is followed by three observable facts about living organisms:
\begin{enumerate}
	\item  tphenotypic variatio - nraits vary among individuals with respect to their morphology,
		physiology and behaviour, 
	\item ddifferential fitnes - sifferent traits confer different rates of survival 
		and reproduction, 
	\item heritability of fitness - traits can be passed from generation to generation.
\end{enumerate}
Thus, in successive generations members of a population are more likely to be replaced by
the progenies of parents with favourable characteristics that have enabled them to survive and 
reproduce in their respective environments. 
In the early 20th century, other competing ideas of evolution such as mutationism and orthogenesis
were refuted as the modern synthesis reconciled Darwinian evolution with classical genetics,
which established adaptive evolution as being caused by natural selection acting on
Mendelian genetic variation.

All life on Earth shares a last universal common ancestor (LUCA) that lived approximately
3.5–3.8 billion years ago.
The fossil record includes a progression from early biogenic graphite, to microbial mat fossils,
to fossilised multicellular organisms.
Existing patterns of biodiversity have been shaped by repeated formations of new species 
(speciation), changes within species (anagenesis) and loss of species (extinction) throughout the 
evolutionary history of life on Earth.
Morphological and biochemical traits are more similar among species that share a more recent 
common ancestor, and can be used to reconstruct phylogenetic trees.

Evolution in organisms occurs through changes in heritable traits—the inherited characteristics of
an organism. 
In humans, for example, eye colour is an inherited characteristic and an individual might inherit 
the "brown-eye trait" from one of their parents.
Inherited traits are controlled by genes and the complete set of genes within an organism's genome
(genetic material) is called its genotype.

The complete set of observable traits that make up the structure and behaviour of an organism is
called its phenotype. 
These traits come from the interaction of its genotype with the environment. 
As a result, many aspects of an organism's phenotype are not inherited. For example, suntanned skin
comes from the interaction between a person's genotype and sunlight; thus, suntans are not passed
on to people's children.
However, some people tan more easily than others, due to differences in genotypic variation;
a striking example are people with the inherited trait of albinism,
who do not tan at all and are very sensitive to sunburn.

Heritable traits are passed from one generation to the next via DNA, a molecule that encodes 
genetic information. DNA is a long biopolymer composed of four types of bases.
The sequence of bases along a particular DNA molecule specify the genetic information,
in a manner similar to a sequence of letters spelling out a sentence.
Before a cell divides, the DNA is copied, so that each of the resulting two cells will inherit the
DNA sequence. Portions of a DNA molecule that specify a single functional unit are called genes;
different genes have different sequences of bases.
Within cells, the long strands of DNA form condensed structures called chromosomes.
The specific location of a DNA sequence within a chromosome is known as a locus.
If the DNA sequence at a locus varies between individuals, the different forms of this sequence
are called alleles. DNA sequences can change through mutations, producing new alleles.
If a mutation occurs within a gene, the new allele may affect the trait that the gene controls,
altering the phenotype of the organism.
However, while this simple correspondence between an allele and a trait works in some cases, 
most traits are more complex and are controlled by quantitative trait loci
(multiple interacting genes).

Recent findings have confirmed important examples of heritable changes that cannot be explained by
changes to the sequence of nucleotides in the DNA.
These phenomena are classed as epigenetic inheritance systems.
DNA methylation marking chromatin, self-sustaining metabolic loops, gene silencing by RNA 
interference and the three-dimensional conformation of proteins (such as prions) are areas where
epigenetic inheritance systems have been discovered at the organismic level. 
Developmental biologists suggest that complex interactions in genetic networks and communication 
among cells can lead to heritable variations that may underlay some of the mechanics in
developmental plasticity and canalisation.
Heritability may also occur at even larger scales. For example, ecological inheritance through
the process of niche construction is defined by the regular and repeated activities
of organisms in their environment. 
This generates a legacy of effects that modify and feed back into the selection regime of 
subsequent generations.
Descendants inherit genes plus environmental characteristics generated by the ecological
actions of ancestors.
Other examples of heritability in evolution that are not under the direct control of genes 
include the inheritance of cultural traits and symbiogenesis. 
%--------------------------------------------------------------------------------------------------
\subsection{Genetics}
Genetics is a branch of biology concerned with the study of genes, genetic variation, and heredity 
in organisms.

Though heredity had been observed for millennia, Gregor Mendel, Moravian scientist and
Augustinian friar working in the 19th century in Brno, was the first to study genetics 
scientifically.
Mendel studied "trait inheritance", patterns in the way traits are handed down from parents to
offspring.
He observed that organisms (pea plants) inherit traits by way of discrete ``units of inheritance''.
This term, still used today, is a somewhat ambiguous definition of what is referred to as a gene.

Trait inheritance and molecular inheritance mechanisms of genes are still primary principles of
genetics in the 21st century, but modern genetics has expanded beyond inheritance to studying the 
function and behavior of genes.
Gene structure and function, variation, and distribution are studied within the context of the
cell, the organism (e.g. dominance), and within the context of a population.
Genetics has given rise to a number of subfields, including molecular genetics,
epigenetics and population genetics. Organisms studied within the broad field span the domains of
life (archaea, bacteria, and eukarya).

Genetic processes work in combination with an organism's environment and experiences to influence
development and behavior, often referred to as nature versus nurture.
The intracellular or extracellular environment of a living cell or organism may switch gene 
transcription on or off. 
A classic example is two seeds of genetically identical corn, one placed in a temperate
climate and one in an arid climate (lacking sufficient waterfall or rain).
While the average height of the two corn stalks may be genetically determined to be equal, 
the one in the arid climate only grows to half the height of the one in the temperate climate
due to lack of water and nutrients in its environment. 


%--------------------------------------------------------------------------------------------------
\subsection{Evolution of intelligence}

%====================================================================================================
\section{Genetic algorithms}

%--------------------------------------------------------------------------------------------------
\subsection{Representation of data}

%--------------------------------------------------------------------------------------------------
\subsection{Selection}

%--------------------------------------------------------------------------------------------------
\subsection{Crossover or asexual reproduction}

%--------------------------------------------------------------------------------------------------
\subsection{Mutation}

%--------------------------------------------------------------------------------------------------
\subsection{Elite}

%====================================================================================================
\section{Neuroevolution}

%--------------------------------------------------------------------------------------------------
\subsection{Deep neural networks}

%--------------------------------------------------------------------------------------------------
\subsection{Neuroevolution for agumented topologies}
