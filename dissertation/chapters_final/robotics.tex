\FloatBarrier
\chapter{Control problems in robotics}
This chapter introduces tasks that are usually preformed by mobile robots and provides short 
overview of approaches, other than one described in this work, used for solving them.
After that challenges related to limited computational resources will be presented as well as 
implementation issues related to time critical nature of control systems.


%==================================================================================================
\FloatBarrier
\section{Sensory information processing}

%--------------------------------------------------------------------------------------------------
\subsection{Sensors}
Receptors, which in case of robots are called sensors, are necessary in order to obtain information
about both external environment and state of robot physical body \cite{Siegwart2011}.i
As there are many different  type of values to measure so there must exists multiple types of
sensors. 
Some sensors are used to measure relatively simple values like the temperature of a robot 
logic processor or the rotational speed of the motors.
On the other hand sensors like a camera receive a very complex information that require
significant processing before any useful information can be extracted from it.
As the mobile robot moves around, it is bound to encounter an unforeseen environmental
situations and therefore ability to gather information about its surrounding is critical.
Sensors can be classified based on two important characteristic: proprioceptive/exteroceptive and
passive/active \cite{Borenstein1997}.
\begin{itemize}
	\item Proprioceptive sensors measure values internal to the system like a motor speed or
		battery voltage,
	\item Exteroceptive sensors acquire information from the robot environment like a distance
		to obstacle, 
	\item Passive sensors measure external signals interacting with them, for example cameras
		that detect light reflected from objects,
	\item Active sensors emit signals into the environment, then measure it reaction, for example
		a sonar will emit ultrasound wave and measure time after which it will return to source. 
\end{itemize}
As active sensors provide a  more controlled interactions with the environment, they  will be
preffered in most cases. However, this approach is not without a downsides:
\begin{itemize}
	\item the outbound energy may affect the very characteristics that the sensor is attempting
		to measure, 
	\item interference between emitted signal and other signal of similar nature, for 
		example generated by other robot, may cause errors.
\end{itemize}
Mobile robots depend heavily on exteroceptive sensors as their role is to acquire information on
objects in the robot’s immediate vicinity so that it may model its environment and make a 
decision on what action to take. All objects are detected within robot local reference 
frame. 
\begin{table}[htb] 
	\centering
	\caption{Example types of sensors used in robotics}
	\label{tab:robot-sensors}
	\begin{tabular}{ccc}
		\hline
		\hline
		Sensor& Type of measurement& Operation mode\\
		\hline
		Bumper&  Exteroceptive& Passive\\
		Encoder& Proprioceptive& Active\\
		Compass& Exteroceptive& Passive\\
		Gyroscope& Proprioceptive& Passive\\
		Beacon localization& Exteroceptive& Passive*\\
		Laser rangefinder& Exteroceptive& Active\\
		CCD/CMOS Camera& Exteroceptive& Passive\\
		\hline
		\hline
	\end{tabular}
	\newline
	{\footnotesize *Beacon system as a whole uses active elements however sensors 
	used on robot are only passive.}
\end{table}
Active ranging sensors tend to have failure modes that are triggered largely by specific 
relative positions of the sensor and environment targets. For example, a sonar sensor will
produce specular reflections, producing grossly inaccurate measurements of range,
at specific angles to a smooth sheetrock wall.  The important point is to realize that, while 
systematic error and random error are well-defined in a controlled setting, the mobile robot can
exhibit error characteristics that bridge the gap between deterministic and stochastic error
mechanisms.



%==================================================================================================
\FloatBarrier
\section{Control systems}

%==================================================================================================
\FloatBarrier
\section{Hardware and software specifications}

%==================================================================================================
\FloatBarrier
\section{Concepts in machine learning}

%--------------------------------------------------------------------------------------------------
\FloatBarrier
\subsection{Differences between knowledge and data based models}
While creating a model of physical processes one of two approaches may be used.
First, is the knowledge-driven model that as the name implies relays on knowledge of underlying
laws of physics that govern the process.
The main advantage of this approach is its high explainability and reliability as every parameter
of the model corresponds with some specific physical property.
However this model has a very significant downside, it requires detailed knowledge of physical
phenomena. This means that for complex processes it might be very difficult or even
impossible, at a~given moment, to create a model. Another drawback is that the complexity of
the process grows the computational cost of its model.
If those problem makes creating usable knowledge-based model impossible or economically
non-efficient a data-driven approach may be used.
It model used is chosen arbitrarily with little or no relation to underlying physics,
then model parameters are adjusted until it will fit experimental data to the best of its
capability.
The main advantage of a data-driven approach is the lack of required knowledge on the process
inner workings, which allow prediction of events which exact mechanics have not
yet been discovered.
Another element characteristic in the data-driven approach, that can be either advantageous or
disadvantageous is the lack of direct correlation between the complexity
of the process and computational cost of the model.
This is a very welcome trait for complex processes like orbital atomic clock ensembles.
The main disadvantage in comparison with the knowledge-driven approach is its 
lower reliability and explainability. 
This is the main reason why the data-driven approach is usually avoided in
critical implementations. Another significant downside is a requirement of a large amount of
experimental data for model adjustment which means that in many cases this approach simply
cannot be used.
The last thing that must be said about the data-driven approach is that it still requires some
amount of knowledge to work efficiently as the selection of models that will be adjusted to
data requires some assumption. For example, the use of linear regression assumes that the model
is linear.

%----------------------------------------------------------------------------------------------------
\FloatBarrier
\subsection{What is a machine learning algorithm}
Unlike artificial intelligence (AI) which is more of a loose term machine learning (ML) is 
more rigidly defined \cite{AurelienGeron2019}.
\begin{definition}[Machine Learning]
Machine learning  describes subset of algorithms that adjusts parameter of other algorithms 
according to data.
\end{definition}
To provide more formal definition an algorithm can be described as a transformation 
function $\phi_{a}$ such that:
\begin{equation}
	\label{equ:algorithm_general}
	\mathcal{Y}_{a} = \phi_{a}(\mathcal{X},\Theta_{a}),
\end{equation}
where $\mathcal{Y}_{a}$ is set of algorithm responses, $\mathcal{X}$ is set of observations
(algorithm inputs) and $\Theta_{a}$ is algorithm parameter set \cite{Tadeusiewicz1994}.
Observation space $\mathcal{U}_{x}$ and response space $\mathcal{U}_{y}$ can be simply a numerical
space, in which case algorithm is called numerical algorithm \cite{G1989}, but can also represent
more abstract concepts. For example in case of database software input space will consist of 
queries while response space will contain information stored in database as well as error 
status in case query was written incorrectly\cite{Banachowski2013}.
Machine learning algorithm, denoted as $\phi_{m}$ can also be described in that general form
however what is special in this case is that input space $\mathcal{U}_{m}$ consists of either :
\begin{equation}
	\label{equ:supervised_input}
	\mathcal{X}_{m} = \{\phi_{a}, \mathcal{X}_{a}, \mathcal{Y}_{a}, \Theta_{a} \},
\end{equation}
in case of supervised learning or just :
\begin{equation}
	\label{equ:supervised_input}
	\mathcal{X}_{m} = \{\phi_{a}, \mathcal{X}_{a}, \Theta_{a} \},
\end{equation}
in case of unsupervised learning.
In both cases response of machine learning algorithm is described as a:
\begin{equation}
	\label{equ:ml_response}
	\mathcal{Y}_{m} = \{\hat{\Theta}_{a}, E_{a}\},
\end{equation}
where $\hat{\Theta}_{a}$ is adjusted set of parameters and $E_{a}$ is value prediction 
error for this new set of parameters.
Aim of machine learning algorithm is to adjust parameters $\Theta_{a}$ of given algorithm 
$\phi_{a}$ in such a way to minimize error $E_{a}$ for given data. In case of supervised learning
a correct response $\mathcal{Y}_{a}$ for inputs is known where in case of unsupervised learning
only input set $\mathcal{X}_{a}$ is known. In order to avoid confusion parameters of machine 
learning algorithm $\Theta_{m}$ will be referred to as a \textit{metaparameters}. 
Exact set of meteaparameters vary between specific algorithms, however there is one 
crucial metaparameter that will appear in every machine learning algorithm.
That metaparameter is error function as it is required for error value calculation and as 
minimization of that value is goal of machine learning it is mandatory to be able to calculate it.
