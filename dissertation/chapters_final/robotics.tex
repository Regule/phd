\FloatBarrier
\chapter{Control problems in robotics}
This chapter introduces tasks that are usually preformed by mobile robots and provides short 
overview of approaches, other than one described in this work, used for solving them.
After that challenges related to limited computational resources will be presented as well as 
implementation issues related to time critical nature of control systems.


%==================================================================================================
\FloatBarrier
\section{Sensory information processing}

%--------------------------------------------------------------------------------------------------
\subsection{Sensors}
Receptors, which in case of robots are called sensors, are necessary in order to obtain information
about both external environment and state of robot physical body \cite{Siegwart2011}.i
As there are many different  type of values to measure so there must exists multiple types of
sensors. 
Some sensors are used to measure relatively simple values like the temperature of a robot 
logic processor or the rotational speed of the motors.
On the other hand sensors like a camera receive a very complex information that require
significant processing before any useful information can be extracted from it.
As the mobile robot moves around, it is bound to encounter an unforeseen environmental
situations and therefore ability to gather information about its surrounding is critical.
Sensors can be classified based on two important characteristic: proprioceptive/exteroceptive and
passive/active \cite{Borenstein1997}.
\begin{itemize}
	\item Proprioceptive sensors measure values internal to the system like a motor speed or
		battery voltage,
	\item Exteroceptive sensors acquire information from the robot environment like a distance
		to obstacle, 
	\item Passive sensors measure external signals interacting with them, for example cameras
		that detect light reflected from objects,
	\item Active sensors emit signals into the environment, then measure it reaction, for example
		a sonar will emit ultrasound wave and measure time after which it will return to source. 
\end{itemize}
As active sensors provide a  more controlled interactions with the environment, they  will be
preffered in most cases. However, this approach is not without a downsides:
\begin{itemize}
	\item the outbound energy may affect the very characteristics that the sensor is attempting
		to measure, 
	\item interference between emitted signal and other signal of similar nature, for 
		example generated by other robot, may cause errors.
\end{itemize}
Mobile robots depend heavily on exteroceptive sensors as their role is to acquire information on
objects in the robot’s immediate vicinity so that it may model its environment and make a 
decision on what action to take. Of course all objects are detected within robot local reference 
frame. 
\begin{table}[htb] 
	\centering
	\caption{Example types of sensors used in robotics}
	\label{tab:robot-sensors}
	\begin{tabular}{ccc}
		\hline
		\hline
		Sensor& Type of measurement& Operation mode\\
		\hline
		Bumper&  Exteroceptive& Passive\\
		Encoder& Proprioceptive& Active\\
		Compass& Exteroceptive& Passive\\
		Gyroscope& Proprioceptive& Passive\\
		Beacon localization& Exteroceptive& Passive*\\
		Laser rangefinder& Exteroceptive& Active\\
		CCD/CMOS Camera& Exteroceptive& Passive\\
		\hline
		\hline
	\end{tabular}
	\newline
	{\footnotesize *Beacon system as a whole uses active elements however sensors 
	used on robot are only passive.}
\end{table}
Active ranging sensors tend to have failure modes that are triggered largely by specific 
relative positions of the sensor and environment targets. For example, a sonar sensor will
produce specular reflections, producing grossly inaccurate measurements of range,
at specific angles to a smooth sheetrock wall.  The important point is to realize that, while 
systematic error and random error are well-defined in a controlled setting, the mobile robot can
exhibit error characteristics that bridge the gap between deterministic and stochastic error
mechanisms.



%==================================================================================================
\FloatBarrier
\section{Control systems}

%==================================================================================================
\FloatBarrier
\section{Hardware and software specifications}
