%==================================================================================================
\FloatBarrier
\chapter{Proposed model}

%==================================================================================================
\FloatBarrier
\section{Device limitations}

%==================================================================================================
\FloatBarrier
\section{Network representations}
The main element for which there is a potential to reduce resource requirements is the node model, 
in this case, two facts have been highlighted. 
First of all, because there are only two objects in the network topology model, it is not required 
to use pointers to links in the node, it is enough to assume that all link objects 
saved after the node belong to it.


As visualized in Figure 6, the connections marked in yellow are right behind their source 
nodes marked in orange. 
This means that there is no need to store information about the number and destination addresses 
of connections because it is enough to iterate through the memory until encountering the next 
node acting as a terminator of the connection string. 
This is similar to the ASCII text string implementation in most operating systems where the 
string is represented by a memory fragment terminated with a character code 0. 
It should be noted that in such a model nodes and links share the same memory area, 
therefore it must be possible to identify the type of an object in memory during program 
execution, knowing only the position of the first byte of such an object in memory. 
In the proposed model it is realized by treating the first bit of the first byte as the flag 
identifying the object type if this bit is set to 0 it is a node and if it is 1 it is a link. 
The second important feature that can be used is the low ratio of active neurons to the total 
number of neurons in the case of larger networks. Since we expect the generated networks to be 
of significant size, we can take advantage of this fact and separate the information about the 
activation of nodes from their models to external objects. 
As a result, the proposed model will consist of three structures: 
\begin{itemize}
	\item node - stores information on the method of aggregation,
		activation and the last cycle in which it was activated,
	\item link - stores the connection weight and the target node address
	\item spider - stores information about the node it is on, about the result of its
		aggregation and then activation, and about the cycle in which it is currently located.
\end{itemize}

The project developed a model representing both a node and a link using a single 32-bit number and 
a spider using two 32-bit numbers. 
This is a significant value because the 32-bit number in fixed-point encoding is a native 
representation of numbers in the processor used in the project. 
Since all the data stored in the model does not match the standard 32-bit number division into 
four bytes, it is not possible to use the structured programming functionality provided by the 
C language. Instead, functions operating on the memory representation as a whole will be used.

As shown in Figure 7, a single 32-bit memory cell is divided into three sections: 
\begin{enumerate}
	\item Flag of type (T) - the highest bit informs about what type of object is represented,
		in case of setting the value to zero it is a node and for one it is a link.
	\item Address/flags (A) - the next eight bits, in the case of a link, act as a shift written
		as an integer in the U2 code, it is an equivalent of the native model pointer.
		For a node, the same eight bits act as flags:
		\begin{enumerate}
				\item Inputs (I) - if set to 1 node is an input.
				\item Outputs (O) - if set to 1 node is an exit.
				\item Aggregation (G) - informs about the type of aggregation function.
				\item Activation (C) - informs about the type of activation function.
				\item Not Used (X) - These bits hold no value. 
		\end{enumerate}
	\item Values (V) - the value stored by the object. For a node, it is the current
		cycle stored as an integer value, while for a connection it is its weight written as a
		fixed-point number with a comma at the 21st-byte position.
\end{enumerate}


%==================================================================================================
\FloatBarrier
\section{Fixed point arithmetics}
In order
