\chapter{Introduction}
This chapter provides information required for understanding basic terminology used in this
work as well as justifies selction of problem as well as methods used.

%==================================================================================================
\FloatBarrier
\section{Hypothesis}
With the application of modern machine learning algorithms, especially artificial neural networks,
it is possible to predict a clock bias with an error of less than 1 $\mu s$ basing only on
previously measured errors. 



%==================================================================================================
\FloatBarrier
\section{Motivation}
I have encountered a problem that was an inspiration for this work during my research in the 
field of mobile robotics.
While my initial focus was directed more towards emergent behavior in robotics and self-organizing 
systems \cite{Gnys2017}\cite{Gnys2019} I have encountered an issue with localization in marine 
robotics \cite{Cabrera-Gamez2014}.
With the high cost of internet connection for robots operating far away from land, 
it is important to be able to calculate precise position without updating data from the internet.
One of the issues when working with a satellite navigation system is a requirement for clock bias
correction, to keep error drift to minimum readouts for both local and satellite clocks must

\FloatBarrier
%==================================================================================================
\section{Terminology used}
When working with multiple sources distributed across different fields of science a problem of 
differences in notation is encountered. To integrate all of gathered information into a single 
coherent work a single notation must be decided on and all equations should be then translated
into that notation.
Due to author interest in cybernetics and multi agent system a terminology with following 
concepts as base will be used:
\begin{enumerate}
	\item system,
	\item observation,
	\item response,
	\item transfer function,
\end{enumerate}
while in specific context different names may be given to specific parts of that model general
concept will remain unchanged.

\begin{definition}[System]
	A system is a set of elements which interact with each other. It can interact with it 
	environment by gathering observations and sending responses.
\end{definition}

\begin{definition}[Observation]
	An observation is a set of information about a state of an environment or about internal
	system state.
\end{definition}

\begin{definition}[Response]
	Response is set of information returned by a system to environment.
\end{definition}

\begin{definition}[Transfer function]
	Transfer function maps an observation into a response.
\end{definition}
In this worke transfer function will be often refered to by a name more associated with its 
current context. For example in case of predictor systems it will be called hypothesis.

be corrected by predicted bias.
