\chapter{Introduction}
This chapter provides information required for understanding basic terminology used in this
work as well as justifies selction of problem as well as methods used.


%==================================================================================================
\FloatBarrier
\section{Motivation}
I have encountered a problem that was an inspiration for this work during my research in the 
field of mobile robotics.
While my initial focus was directed more towards emergent behavior in robotics and self-organizing 
systems \cite{Gnys2017} \cite{Gnys2019} I have encountered an issue with localization in marine 
robotics \cite{Cabrera-Gamez2014}.
With the high cost of internet connection for robots operating far away from land, 
it is important to be able to calculate precise position without updating data from the internet.
One of the issues when working with a satellite navigation system is a requirement for clock bias
correction, to keep error drift to minimum readouts for both local and satellite clocks must

\FloatBarrier
%==================================================================================================
\section{Goals and scope of the work}

\FloatBarrier
%==================================================================================================
\section{Related work}


\FloatBarrier
%==================================================================================================
\section{Summary of contributions}
Whithin this work a two conributions were made. First and foremost a representation model of
neural network that is cappable of working on limited resource platform was proposed.
After that a NEAT based genetic algorithm that is cappable of creating networks capable of being
represented in aftermentioned model was designed and implemented.

\FloatBarrier
%==================================================================================================
\section{GPL Licensing}
As some of the models and tools used in the project is based on the GPL license, it is necessary 
to understand the legal consequences of this. 
The purpose of the GNU GPL license is to give users four basic freedoms: 
\begin{enumerate}
\item Freedom to run the program for any purpose 
\item Freedom to analyze how the program works and to adapt it to your needs 
\item Freedom to distribute the unmodified copy of the program 
\item Freedom to improve the program and to disseminate your improvements to the public, so that 
	the whole community can benefit from them.i
\end{enumerate}
Since any work based on a work covered by a GPL must also be based on that license, the GPL 
is a viral license. 
For this reason, the software developed as part of the project and being a modification of the 
existing NEAT-based solutions must implement the freedoms resulting from the GPL license. 
This license, however, does not pass to products made using the GPL software, but only to its 
modifications. 
A good example of this can be a GLP-based text editor, its modified version must have the same 
license, while texts written with the editor may be subject to different licensing.
