%==================================================================================================
\chapter{Neural networks}
\FloatBarrier

In this chapter attention will be given to artificial neural networks (ANN) which are one of most 
popular data processing models in self learning systems \cite{Abiodun2019} \cite{Tran2021}
\cite{Syed2021}. First focus will be given to biological neural networks, that will be reffered 
to as a neural circuits (NC) in order to be consistent with nomenclature \cite{Purves2001} as 
well as to avoid confusion with ANN.  
After that an electrical models of NC will be shortly described and finally a mathematical models
that are used in modern ANN solutions will be discussed. Special attention will be given to 
computational complexity of both running and teaching described ANN models. 



%==================================================================================================
\section{Neural circuits in biological systems}
\FloatBarrier
In case of organic systems two kinds of information processing subsystems can be distinguished
\cite{Johnson2016}:
\begin{itemize}
	\item neural system which is low latency localized signal processing system that requires
	both source and sink of signal to be directly connected to it, 
	\item endocrine system which is high latency system that influences whole organism and do 
	not require direct connection to affected cells as information is delivered in form of 
	hormones in blood.
\end{itemize}
While both systems interact witch each other and are important for correct functioning of 
organisms \cite{Schwarz2019} in this work attention will be given only to neural system with
special focus on interneurons that process already collected data.


%--------------------------------------------------------------------------------------------------
\subsection{Neural cell as a base of information processing circuit}
\FloatBarrier
At most basic level a neuron is an animal cell which means
