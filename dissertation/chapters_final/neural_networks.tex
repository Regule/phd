%==================================================================================================
\FloatBarrier
\chapter{Neural networks}

In this chapter attention will be given to artificial neural networks (ANN) which are one of most 
popular data processing models in self learning systems \cite{Abiodun2019} \cite{Tran2021}
\cite{Syed2021}. First focus will be given to biological neural networks, that will be referred 
to as a neural circuits (NC) in order to be consistent with nomenclature \cite{Purves2001} as 
well as to avoid confusion with ANN.  
After that an electrical models of NC will be shortly described and finally a mathematical models
that are used in modern ANN solutions will be discussed. Special attention will be given to 
computational complexity of both running and teaching described ANN models. 



%==================================================================================================
\FloatBarrier
\section{Basic concepts}

%--------------------------------------------------------------------------------------------------
\FloatBarrier
\subsection{Biological networks}
The subject of artificial neural networks belongs to the interdisciplinary field of research 
related to biocybernetics, electronics, applied mathematics, automation, and even medicine.
Artificial neural networks were created based on knowledge about the functioning of the nervous
system of living creatures and are an attempt to use the phenomena occurring in nervous systems
in the search for new technological solutions.
The nerve cell, or neuron for short, is the basic component of the nervous system.
Understanding the mechanisms of operation of individual neurons and their interaction is 
particularly important for understanding the processes of acquiring, transmitting, processing,
and using information running in neural networks.
For this reason, the real neuron model is extremely important. Like any other cell, a neuron has
a body with cytological equipment called a soma, inside which is a nucleus.
From the soma of the neuron - numerous projections play an important role in connection with
other cells.
Two types of projections can be distinguished: numerous, thin, and densely branched dendrites
and a thicker axon, forking at the end.
Input signals go to the cell via synapses, and the output signal goes through the axon
and its numerous branches called collateral.
The collaterals reach the soma and dendrites of other neurons to form more synapses. 
Synapses that connect the outputs of other nerve cells to a given cell can therefore be found
both on the dendrites and directly on the cell body.
The transmission of signals inside the nervous system is a complex chemical-electrical process. 
In simplified terms, it can be assumed that the transmission of the nerve impulse from one cell to
another is based on the secretion of special chemical substances under the influence of stimuli 
coming from synapses.

These substances act on the cell membrane to cause a change in its electric potential, 
the change being stronger the more neuro mediators appear on the cell membrane.
Individual synapses differ in size and the ability to accumulate neuro mediators near the synaptic
membrane. 
For this reason, the same impulse reaching the cell's input via a specific synapse may cause a
stronger or weaker excitation of the cell than in the case of another input.
The degree of cell excitation is measured by the degree of its membrane polarization,
which depends on the total amount of neuro mediators emitted in all synapses.
It follows that the cell inputs can be assigned numerical coefficients (weights) corresponding to 
the number of neuro mediators isolated at one time on individual synapses.
In the mathematical model, input signals must be multiplied by these coefficients to correctly 
take into account the influence of individual input signals on the state of the nerve cell.
Synaptic weights are real numbers and can take both positive and negative values.
One of them has a stimulating effect and the other is inhibitory, making it difficult for the cell
to be stimulated by other signals.
The action of the excitatory synapse can be interpreted as assuming a positive value for 
the synaptic weight, and a negative value for the inhibitory synapse.
As a result of the input pulses reaching individual synapses and the release of appropriate
amounts of the neuro mediators, specific electrical stimulation of the cell takes place.
If the electrical imbalance is minor or if the balance of excitations and inhibitions is negative,
the cell returns to its initial state by itself, and no change can be seen in its output.

If, on the other hand, the sum of stimulations and inhibitions exceeds the cell activation
threshold, the output signal increases rapidly and a characteristic shape of
the nerve impulse is formed, sent by an axon to other neurons connected to a given cell.
This signal depends on the degree of exceeding the threshold.
The cell works on the principle: all or nothing. After fulfilling his role, the neuro mediators
are removed. The mechanism of its removal is either absorbed by the cell,
broken down, or moved beyond the synaptic area.
It stays at the same time as generating an impulse by the nerve cell the refraction
process is running.
It is a rapid increase in the cell activation threshold to the infinite value, 
as a result of which, immediately after generating the impulse, the neuron will not be able 
to generate the next one, even with strong stimulation. This state persists for some time Ar,
called absolute refraction.
This period is followed by a period of relative refraction, delta t, in which the operating
threshold returns to its resting value. In the period of At.
the cell can be stimulated, but with the use of stronger forces.
Usually in real processes the following relation Ar »At is fulfilled.

%==================================================================================================
\FloatBarrier
\section{Representation in computer}

%==================================================================================================
\FloatBarrier
\section{Learining algorithms}

%==================================================================================================
\FloatBarrier
\section{Neuroevolution}
